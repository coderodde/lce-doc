\documentclass[10pt]{article}

\usepackage[english]{babel}
\usepackage[utf8x]{inputenc}
\usepackage{amsmath}
\usepackage{amssymb}
\usepackage{amsfonts}
\usepackage{graphicx}
\usepackage[ruled]{algorithm2e}
\usepackage{empheq}
\usepackage{float}
\usepackage{tikz}

\DeclareMathOperator*{\argmin}{\arg\,\min}

\title{Computing debt cuts leading to global zero-equity}
\author{Rodion ``rodde'' Efremov}

\begin{document}
  \maketitle
  
  \begin{abstract}
  In this paper we present a method for computing a set of loan cuts, which, once applied, lead to a global zero-equity state, i.e., each and every party in the financial network may forget all liabilities.
  \end{abstract}
  
  \section{Basic definitions}
  Before we proceed to defining the structures needed in discussing the method, we have to impose some definitions: by $\mathfrak{R}_?$, we denote the set of real numbers $x$ such that $x \, ? \, 0$ holds. We work on a dynamic graph $G=(V,A)$, $A \subset V \times V$, for which we define a weight function $w_G \colon V^2 \to \mathcal{P}(\mathfrak{R}_> \times \mathfrak{R}_{\geq} \times \mathfrak{R}_> \times \mathfrak{R})$ 
\end{document}