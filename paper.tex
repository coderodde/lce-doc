\documentclass[10pt]{article}

\usepackage[english]{babel}
\usepackage[utf8x]{inputenc}
\usepackage{amsmath}
\usepackage{amssymb}
\usepackage{amsfonts}
\usepackage{graphicx}
\usepackage[ruled]{algorithm2e}
\usepackage{empheq}
\usepackage{float}

\newcommand{\RR}{\mathfrak{R}}
\newcommand{\KK}{\mathfrak{K}}
\newcommand{\CC}{\mathfrak{C}}
\newcommand{\ff}{\mathfrak{f}}

\newtheorem{mydef}{Definition}

\title{Computing debt cuts leading to global zero-equity}
\author{Rodion ``rodde'' Efremov}

\begin{document}
  \maketitle
  
  \begin{abstract}
  In this paper we present a method for computing a set of loan cuts, which, once applied, lead to a global zero-equity state, i.e., each and every party in a financial network may forget all liabilities. 
  \end{abstract}
  
  \section{Basic definitions}
  Before we proceed to defining the structures needed in discussing the method, we have to impose some definitions: by $\RR_?$, we denote the set of real numbers $x$ such that $x \, ? \, 0$ holds. We work on a directed graph $G=(V,A)$, $A \subset V \times V$, for which we define a weight function $w_G \colon G.A \to \mathcal{P}(\RR_> \times \RR_{\geq} \times (\RR_> \cup \{\infty\}) \times \RR)$. If $(K, r, n, t) \in \mathcal{P}(\RR_> \times \RR_{\geq} \times (\RR_> \cup \{\infty\}) \times \RR)$, $K$ is the initial principal investment, $r$ is the annual interest rate, $n$ the amount of compounding periods per year (the value of $\infty$ is allowed, which denotes continuous compounding), and $t$ is the time point at which the loan was granted. Together the four parameters comprise a \textit{contract}. (Note that the ``weight'' of an edge is a \textbf{set} of contracts as this is physically possible in real-world banking.) 

The most fundamental function in this paper is the equity function $e_G \colon G.V \times \RR \to \RR$ defined as 
\begin{align*}
e_G(u, \tau) &= \sum_{(u, v) \in G.A} \Bigg( \sum_{(K, r, n, t) \in w_G(u, v)} \CC_{\tau}(K, r, n, t) \Bigg) \\
&- \sum_{(v, u) \in G.A} \Bigg( \sum_{(K, r, n, t) \in w_G(v,u)} \CC_{\tau}(K, r, n, t) \Bigg),
\end{align*}
where 
\[
\CC_{\tau}(K, r, n, t) = 
\begin{cases}
K\big( 1 + \frac{r}{n}\big)^{\lfloor n(\tau - t) \rfloor} & \mbox{if } n \in \mathfrak{N} \\
Ke^{r(\tau - t)} & \mbox{if } n = \infty,
\end{cases}
\]
and $\tau$ is no less than the time stamp of the most recent loan of $u$. Also, we assume we are given a function $\ff_G \colon G.V \to \RR$ mapping every party $u$ in the financial graph to a time point at which $u$ is ready to pay back at most all of its debts. (By ``at most'' we mean that we will try to minimize the magnitude of the nodes' debt cuts, yet it is not possible for a node, say $v$, having no loans to the other nodes, to have a debt cut any less than the sum of its loans at $\ff_G(v)$, which implies that such node $v$ will have to pay all debts at once.)

As the concept of equilibria in this paper is a global phenomenon, we demand that a time point $T_G$ is given; especially, we demand that $T_G \geq \max_{u \in G.V} \{ \ff_G(u) \}$. Whenever a party, say $u \in G.V$, is ready to raise $C$ amount of resources for the debt cut to $v$ (which is supposed to happen at $\ff(u)$), with $(K, r, n, t) \in w_G(v, u)$, the contract being cut becomes 
\[
\CC_{\tau}\big(\CC_{\ff_G(u)}(K, r, n, t) - C, r, n, \ff_G(u) \big),
\]
$\tau \geq \ff_G(u)$.
 Next we define the concept of equilibrium.
\begin{mydef}
The financial graph $G$ is said to be in equilibrium at time point $\tau$ if and only if $e_G(u,\tau) = 0$ for all $u \in G.V$.
\end{mydef}
Once given $G$, $\ff_G$ and $T_G$, we attempt to compute a function $\Xi_G \colon G.A \times \RR_> \times \RR_{\geq} \times (\RR_> \cup \{ \infty \}) \times \RR \to \RR_{\geq}$ such that after applying a debt cut from $v$ to $u$ of magnitude $\Xi(u, v, \KK)$ against the contract $\KK$ at time point $\ff_G(v)$ for  all $(u, v) \in G.A$, $G$ obtains such a state that it evolves towards equilibrium at time point $T_G$.

\section{Solution}
Whenever a node $u$ has incoming contracts from a set of parent nodes (lenders) $L$, outgoing contracts to a set of children (debtors) $D$, the equilibrium equation for $u$ is
\begin{equation*}
\begin{aligned}
\sum_{v \in D} \sum_{\KK \in w_G(u, v)} \CC_{T_G} \big( \CC_{\ff_G(v)} ( \KK.K, \KK.r, \KK.n, \KK.t ) - \Xi(u, v, \KK), \KK.r, \KK.n, \ff_G(v) \big ) & - \\ 
\sum_{v \in L} \sum_{\KK \in w_G(v, u)} \CC_{T_G} \big( \CC_{\ff_G(u)} ( \KK.K, \KK.r, \KK.n, \KK.t ) - \Xi(v, u, \KK), \KK.r, \KK.n, \ff_G(u) \big) & = 0.
\end{aligned}
\end{equation*}
The above equation is equivalent to
\begin{equation}
\label{eq:equilibrium}
\begin{aligned}
& \sum_{v \in D} \sum_{\KK \in w_G(u, v)} \CC_{T_G} \big( \Xi(u, v, \KK), \KK.r, \KK.n, \ff_G(v) \big) - \\
& \sum_{v \in L} \sum_{\KK \in w_G(v, u)} \CC_{T_G} \big( \Xi(v, u, \KK), \KK.r, \KK.n, \ff_G(u) \big) = \\
& \sum_{ v \in D } \sum_{ \KK \in w_G(u, v) } \CC_{T_G} \big( \CC_{ \ff_G(v) } (\KK.K, \KK.r, \KK.n, \KK.t), \KK.r, \KK.n, \ff_G( v ) \big) - \\
& \sum_{ v \in L } \sum_{ \KK \in w_G(v, u) } \CC_{T_G} \big( \CC_{ \ff_G(u) } (\KK.K, \KK.r, \KK.n, \KK.t), \KK.r, \KK.n, \ff_G( u ) \big). \\
\end{aligned}
\end{equation}
Now if we write down equilibrium equations for all nodes $v \in G.V$, we obtain a linear system, which is guaranteed to have a solution as we can choose for each $\KK \in w_G(u, v)$ a debt cut of magnitude $\CC_{\ff_G(v)}(\KK)$, which results in trivial equilibrium. All the terms $\Xi(\star, \star, \star)$ are to be determined, yet everything else in \eqref{eq:equilibrium} are known before-hand. 

Next, we rewrite all equilibrium equations as a $\RR^{|G.V| \times C'_G}$ - matrix, where 
\[
C'_G = 1 + \sum_{(u, v) \in G.A} |w_G(u, v)|.
\]
(We add 1 as the matrix is supposed to be augmented.) Having the matrix filled with entries, we reduce it to reduced row echelon form, which has at least one solution (the trivial one).
We will obtain two sets: a set of independent variables $S_i = \{ s_1, \dots, s_n \}$ and a set of dependent variables $S_d = \{ s'_1, \dots, s'_m \colon s'_i = f_i(s_1, \dots, s_n) \text{ for some linear } f_i \}$. (In case of finite solution set containing only trivial solution, we have $|S_d| = 0$; in such a case, $S_i$ will contain all the debt cuts.) In case of infinite solution set we choose one that minimizes
\[
\sum_{i = 1}^{n} s_i + \sum_{i = 1}^{m} s'_i = \sum_{i = 1}^{n} s_i + \sum_{i = 1}^{m} f_i(s_1, \dots, s_n) 
\]
subject to constraint of not exceeding a contract with a debt cut, which is a linear program that, once solved, yields the desired $\Xi$.

\section{Experimental results}
Here goes the experiments...

\end{document}