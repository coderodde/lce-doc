\documentclass[10pt]{article}

\usepackage[english]{babel}
\usepackage[utf8x]{inputenc}
\usepackage{amsmath}
\usepackage{amssymb}
\usepackage{amsfonts}
\usepackage{graphicx}
\usepackage[ruled]{algorithm2e}
\usepackage{empheq}
\usepackage{float}

\newcommand{\RR}{\mathfrak{R}}
\newcommand{\KK}{\mathfrak{K}}
\newcommand{\CC}{\mathfrak{C}}
\newcommand{\ff}{\mathfrak{f}}

\title{Computing debt cuts leading to global zero-equity - example}

\begin{document}
  \maketitle

\begin{center}
\centerline{\includegraphics{lce-sample}}
\end{center}
\begin{description}
\item[ab] $(3,0.14,\infty,1)$,
\item[bc] $(2, 0.11, 4, 1.4)$,
\item[ca] $(2.5, 0.05, 6, 0.5)$,
\item[ca2] $(1, 0.27, \infty, 1.7)$.
\end{description}
Equilibrium equation for node $A$:
\begin{align*}
\Xi(ab)e^{0.14 (11.7 - 6)} - \Xi(ca) \Big( 1 + \frac{0.05}{6} \Big)^{ \lfloor 6 (11.7 - 4) \rfloor } - \Xi(ca2) e^{0.27 (11.7 - 4)} &= \\
\underline{2.221 \, \Xi(ab) - 1.464 \, \Xi(ca) - 7.996 \, \Xi(ca2)} &= \\
\CC_{T_G} ( 3e^{ 0.14(6 - 1) }, 0.14, \infty, 6 ) &- \\
\CC_{T_G} ( 2.5 \Big( 1 + \frac{0.05}{6} \Big)^{ \lfloor 6(4 - 0.5) \rfloor }, 0.05, 6, 4 ) &- \\
 \CC_{T_G} ( e^{ 0.27 (4 - 1.7) }, 0.27, \infty, 4 ) &= \\
 \CC_{T_G} (6.041, 0.14, \infty, 6) - \CC_{T_G} ( 2.975, 0.05, 6, 4 ) - \CC_{T_G} ( 1.861, 0.27, \infty, 4 ) &= \\
 13.417 - 4.358 - 14.881 = \underline{-5.822}.
\end{align*}
Equilibrium equation for node $B$:
\begin{align*}
\Xi(bc) \Big( 1 + \frac{0.11}{4} \Big)^{ \lfloor 4(11.7 - 7) \rfloor } - \Xi(ab) e^{0.14( 11.7 - 6 ) } &= \\
\underline{ 1.629 \, \Xi(bc) - 2.221 \, \Xi(ab) } &= \\
\CC_{T_G} (2 \Big( 1 + \frac{0.11}{4} \Big)^{ \lfloor 4(7 - 1.4) \rfloor }, 0.11, 4, 7 ) - \CC_{T_G} ( 3e^{0.14 ( 6 - 1 ) }, 0.14, \infty, 6 ) &= \\
\CC_{T_G} ( 3.632, 0.11, 4, 7 ) - \CC_{T_G} ( 6.041, 0.14, \infty, 6 ) &= \\ 
3.632 \Big( 1 + \frac{0.11}{4} \Big)^{\lfloor 4(11.7 - 7) \rfloor} - 6.041e^{0.14(11.7 - 6)} &= \\
5.918 - 13.417 &= \\
\underline{-7.499}
\end{align*}
Equilibrium equation for node $C$:
\begin{align*}
\Xi(ca) \Big( 1 + \frac{0.05}{6} \Big)^{ \lfloor 6 ( 11.7 - 4) \rfloor } + \Xi(ca2) e^{ 0.27(11.7 - 4) } - \Xi(bc) \Big( 1 + \frac{0.11}{4} \Big)^{ \lfloor 4 (11.7 - 7) \rfloor } &= \\
\underline{1.464 \, \Xi(ca) +7.996 \, \Xi(ca2) - 1.629 \, \Xi(bc)} &= \\
\CC_{T_G} ( 2.5 \Big( 1 + \frac{0.05}{6} \Big)^{ \lfloor 6 ( 4 - 0.5 ) \rfloor }, 0.05, 6, 4) + \CC_{T_G} ( e^{ 0.27 (4 - 1.7) }, 0.27, \infty, 4 ) &- \\
\CC_{T_G} ( 2 \Big( 1 + \frac{0.11}{4} \Big)^{ 4 (7 - 1.4) }, 0.11, 4, 7 ) &= \\
\CC_{T_G} (2.976, 0.05, 6, 4) + \CC_{T_G} ( 1.861, 0.27, \infty, 4 ) - \CC_{T_G} ( 3.633, 0.11, 4, 7 ) &= \\
4.359 + 14.881 - 5.920 &= \\
\underline{13.32}
\end{align*}
Next, the matrix:
\[
\begin{bmatrix}
2.221 & 0 & -1.464 & -7.996 & -5.822 \\
-2.221 & 1.629 & 0 & 0 & -7.499 \\
0 & -1.629 & 1.464 & 7.996 & 13.32 \\
\end{bmatrix}.
\]
Add the first row to the second one:
\[
\begin{bmatrix}
2.221 & 0 & -1.464 & -7.996 & -5.822 \\
0 & 1.629 & -1.464 & -7.996 & -13.32 \\
0 & -1.629 & 1.464 & 7.996 & 13.32 \\
\end{bmatrix}.
\]
Now add the second row to the third:
\[
\begin{bmatrix}
2.221 & 0 & -1.464 & -7.996 & -5.822 \\
0 & 1.629 & -1.464 & -7.996 & -13.32 \\
0 & 0 & 0 & 0 & 0 \\
\end{bmatrix}.
\]
Divide the 1st row by $2.221$:
\[
\begin{bmatrix}
1 & 0 & -0.659 & -3.600 & -2.621 \\
0 & 1.629 & -1.464 & -7.996 & -13.32 \\
0 & 0 & 0 & 0 & 0 \\
\end{bmatrix}.
\]
Divide the 2nd row by $1.629$:
\[
\begin{bmatrix}
1 & 0 & -0.659 & -3.600 & -2.621 \\
0 & 1 & -0.899 & -4.908 & -8.177 \\
0 & 0 & 0 & 0 & 0 \\
\end{bmatrix}.
\]
Now, $\Xi(ca), \Xi(ca2)$ are the independent variables, and 
\begin{align*}
\Xi(ab) &= -2.621 + 0.659 \, \Xi(ca) + 3.600 \, \Xi(ca2) \\
\Xi(bc) &= -8.177 + 0.899 \, \Xi(ca) + 4.908 \, \Xi(ca2).
\end{align*}
Also,
\begin{align*}
0 & \leq \Xi(ca) \leq 2.5 \Big( 1 + \frac{0.05}{6} \Big)^{ \lfloor 6 (4 - 0.5) \rfloor } & = 2.976 \\
0 & \leq \Xi(ca2) \leq e^{0.27 (4 - 1.7)} & = 1.861 \\
0 & \leq \Xi(ab) \leq 3e^{0.14(6 - 1)} &= 6.041 \\
0 & \leq \Xi(bc) \leq 2 \Big( 1 + \frac{0.11}{4} \Big)^{ \lfloor 4(7 - 1.4) \rfloor } &= 3.535
\end{align*}
The last two inequalities are equivalent to 
\begin{align*}
0 \leq & -2.621 + 0.659 \Xi(ca) +3.600 \Xi(ca2) & \leq 6.041 \\
0 \leq & -8.177 + 0.899 \Xi(ca) + 4.908 \Xi(ca2) & \leq 3.535
\end{align*}
Now, we obtain:
\begin{align*}
2.621 \leq 0.659 \Xi(ca) + 3.600 \Xi(ca2) \leq 8.662 \\
8.177 \leq 0.899 \Xi(ca) + 4.908 \Xi(ca2) \leq 11.712
\end{align*}
The objective function is 
\[
\Xi(ab) + \Xi(bc) + \Xi(ca) + \Xi(ca2) = 2.558 \, \Xi(ca) + 9.508 \, \Xi(ca2).
\]
The optimal solution is $\Xi(ca) = 0$, $\Xi(ca2) = 1.665$ so the cuts are:
\begin{align*}
\Xi(ab) &= 3.373\\
\Xi(bc) &= 0\\
\Xi(ca) &= 0\\
\Xi(ca2) &= 1.665
\end{align*}
\end{document}